{\centering{\bfseries{ВВЕДЕНИЕ}}\par}
\addcontentsline{toc}{section}{ВВЕДЕНИЕ}

Развитие области электронного документооборота (ЭДО) является одной из приоритетных и актуальных задач цифровизации государственных и частных секторов, об этом свидетельствует ежегодно увеличивающийся на 15\% объём электронного документооборота\cite{dadaya}. 

Системы электронного документооборота (СЭД) должны отвечать высоким стандартам в области безопасности, эффективности и надёжности, где критическим показателем является своевременное исполнение процессов\cite{dadaya}. В связи с этим основное внимание в области ЭДО уделяется контролю процессов управления, нежели управлению потоком данных\cite{dadaya}. CЭД должна регулярно проходить проверки на соответствие стандартам и требованиям проектирования соответствующих систем\cite{dadaya}. 

Целью данной работы является анализ существующих методов моделирования процессов цифрового документооборота, включая сети Петри и квантовое моделирование, а также разработка системы, способной проводить данное моделирование. Для достижения поставленной цели необходимо решить следующие задачи:
\begin{itemize}[label=---]
	\item проанализировать и проклассифицировать методы моделирования процессов цифрового документооборота с использованием различных сетей Петри;
	\item проанализировать и проклассифицировать методы квантового моделирования процессов цифрового документооборота;
	\item сформулировать общие требования к разрабатываемой системе;
	\item разработать алгоритмы, позволяющие промоделировать процессы цифрового документооборота;
	\item промоделировать процессы цифрового документооборота с использованием разработанного программного обеспечения;
	\item проанализировать полученные результаты.
\end{itemize}

\clearpage