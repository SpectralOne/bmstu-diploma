%----------------------- Преамбула -----------------------
\documentclass[ut8x, 14pt, oneside, a4paper]{extarticle}

\usepackage{extsizes} % Для добавления в параметры класса документа 14pt

% Для работы с несколькими языками и шрифтом Times New Roman по-умолчанию
\usepackage[english,russian]{babel}
\usepackage[T1, T2A]{fontenc}
\usepackage{fontspec}
\setmainfont{Times New Roman}
\usepackage[left=30mm,right=10mm,top=20mm,bottom=20mm]{geometry}
\usepackage{misccorr}
\usepackage{indentfirst}
\usepackage{enumitem}
%\usepackage{ragged2e}
\setlength{\parindent}{1.25cm}
\renewcommand{\baselinestretch}{1.5}
\setlist{nolistsep} % Отсутствие отступов между элементами \enumerate и \itemize

% Дополнительное окружения для подписей
\usepackage{array}
\newenvironment{signstabular}[1][1]{
	\renewcommand*{\arraystretch}{#1}
	\tabular
}{
	\endtabular
}
\usepackage{changepage}

% Переопределение стандартных \section, \subsection, \subsubsection по ГОСТу;
% Переопределение их отступов до и после для 1.5 интервала во всем документе
\usepackage{titlesec}

\titleformat{\section}[block]
{\bfseries\normalsize\filcenter}{\thesection}{1em}{}

\titleformat{\subsection}[hang]
{\bfseries\normalsize}{\thesubsection}{1em}{}
\titlespacing\subsection{\parindent}{\parskip}{\parskip}

\titleformat{\subsubsection}[hang]
{\bfseries\normalsize}{\thesubsubsection}{1em}{}
\titlespacing\subsubsection{\parindent}{\parskip}{\parskip}

\newcommand{\specsection}[1]{\section*{#1}\addcontentsline{toc}{section}{#1}}

% Работа с изображениями и таблицами; переопределение названий по ГОСТу
\usepackage{caption}
\captionsetup[figure]{name={Рисунок},labelsep=endash}
\captionsetup[table]{singlelinecheck=false, labelsep=endash}

\usepackage{graphicx}
\usepackage{adjustbox}
\usepackage{diagbox} % Диагональное разделение первой ячейки в таблицах

% Цвета для гиперссылок и листингов
\usepackage{color}


% Гиперссылки \toc с кликабельностью
\usepackage[linktoc=all]{hyperref}
\hypersetup{hidelinks}


\usepackage{color} % Цвета для гиперссылок и листингов
\hypersetup{citecolor=black}
\newcommand\nostringstyle[1]{\if@instring\else#1\fi}
\usepackage{listings}
\lstset{
	basicstyle=\footnotesize\ttfamily,
	language=prolog,
	numbers=left,
	numbersep=5pt,
	tabsize=2,
	extendedchars=true,
	breaklines=true,
	keywordstyle=\color{blue},
	frame=single,
	showspaces=false,
	showtabs=false,
	xleftmargin=17pt,
	framexleftmargin=17pt,
	framexrightmargin=-5pt,
	framexbottommargin=4pt,
	showstringspaces=false,
	inputencoding=utf8,
	keepspaces=true,
	texcl=true,
	literate={Ö}{{\"O}}1
	{Ä}{{\"A}}1
	{Ü}{{\"U}}1
	{ß}{{\ss}}1
	{ü}{{\"u}}1
	{ä}{{\"a}}1
	{ö}{{\"o}}1
	{~}{{\textasciitilde}}1
	{а}{{\selectfont\char224}}1
	{б}{{\selectfont\char225}}1
	{в}{{\selectfont\char226}}1
	{г}{{\selectfont\char227}}1
	{д}{{\selectfont\char228}}1
	{е}{{\selectfont\char229}}1
	{ё}{{\"e}}1
	{ж}{{\selectfont\char230}}1
	{з}{{\selectfont\char231}}1
	{и}{{\selectfont\char232}}1
	{й}{{\selectfont\char233}}1
	{к}{{\selectfont\char234}}1
	{л}{{\selectfont\char235}}1
	{м}{{\selectfont\char236}}1
	{н}{{\selectfont\char237}}1
	{о}{{\selectfont\char238}}1
	{п}{{\selectfont\char239}}1
	{р}{{\selectfont\char240}}1
	{с}{{\selectfont\char241}}1
	{т}{{\selectfont\char242}}1
	{у}{{\selectfont\char243}}1
	{ф}{{\selectfont\char244}}1
	{х}{{\selectfont\char245}}1
	{ц}{{\selectfont\char246}}1
	{ч}{{\selectfont\char247}}1
	{ш}{{\selectfont\char248}}1
	{щ}{{\selectfont\char249}}1
	{ъ}{{\selectfont\char250}}1
	{ы}{{\selectfont\char251}}1
	{ь}{{\selectfont\char252}}1
	{э}{{\selectfont\char253}}1
	{ю}{{\selectfont\char254}}1
	{я}{{\selectfont\char255}}1
	{А}{{\selectfont\char192}}1
	{Б}{{\selectfont\char193}}1
	{В}{{\selectfont\char194}}1
	{Г}{{\selectfont\char195}}1
	{Д}{{\selectfont\char196}}1
	{Е}{{\selectfont\char197}}1
	{Ё}{{\"E}}1
	{Ж}{{\selectfont\char198}}1
	{З}{{\selectfont\char199}}1
	{И}{{\selectfont\char200}}1
	{Й}{{\selectfont\char201}}1
	{К}{{\selectfont\char202}}1
	{Л}{{\selectfont\char203}}1
	{М}{{\selectfont\char204}}1
	{Н}{{\selectfont\char205}}1
	{О}{{\selectfont\char206}}1
	{П}{{\selectfont\char207}}1
	{Р}{{\selectfont\char208}}1
	{С}{{\selectfont\char209}}1
	{Т}{{\selectfont\char210}}1
	{У}{{\selectfont\char211}}1
	{Ф}{{\selectfont\char212}}1
	{Х}{{\selectfont\char213}}1
	{Ц}{{\selectfont\char214}}1
	{Ч}{{\selectfont\char215}}1
	{Ш}{{\selectfont\char216}}1
	{Щ}{{\selectfont\char217}}1
	{Ъ}{{\selectfont\char218}}1
	{Ы}{{\selectfont\char219}}1
	{Ь}{{\selectfont\char220}}1
	{Э}{{\selectfont\char221}}1
	{Ю}{{\selectfont\char222}}1
	{Я}{{\selectfont\char223}}1
}

\usepackage{ulem} % Нормальное нижнее подчеркивание
\usepackage[figure,table]{totalcount} % Подсчет изображений, таблиц
\usepackage{rotating} % Поворот изображения вместе с названием
\usepackage{lastpage} % Для подсчета числа страниц

\makeatletter
\renewcommand\@biblabel[1]{#1.}
\makeatother

\newcommand{\anonsection}[1]{%
	\section*{#1}%
	\addcontentsline{toc}{section}{#1}%
}

\usepackage{amsmath}
\usepackage{slashbox}
\usepackage{pgfplots}
\pgfplotsset{width=0.9\linewidth,compat=1.18} 