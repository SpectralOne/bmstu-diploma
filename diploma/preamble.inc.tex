%----------------------- Преамбула -----------------------
\documentclass[ut8x, 14pt, oneside, a4paper]{extarticle}

\usepackage{extsizes} % Для добавления в параметры класса документа 14pt

% Для работы с несколькими языками и шрифтом Times New Roman по-умолчанию
\usepackage[english,russian]{babel}
\usepackage[T1, T2A]{fontenc}
\usepackage{fontspec}
\setmainfont{Times New Roman}

%\newfontfamily\cyrillicfont[Script=Cyrillic]{Times New Roman}
%\newfontfamily\cyrillicfontsf[Script=Cyrillic]{Times New Roman}
%\newfontfamily\cyrillicfonttt[Script=Cyrillic]{Times New Roman}
%\usepackage{polyglossia}
%\setdefaultlanguage{russian}

\usepackage[left=30mm,right=10mm,top=20mm,bottom=20mm]{geometry}
\usepackage{misccorr}
\usepackage{indentfirst}
\usepackage{enumitem}
%\usepackage{ragged2e}
\setlength{\parindent}{1.25cm}
\renewcommand{\baselinestretch}{1.5}
\setlist{nolistsep} % Отсутствие отступов между элементами \enumerate и \itemize

% Дополнительное окружения для подписей
\usepackage{array}
\newenvironment{signstabular}[1][1]{
	\renewcommand*{\arraystretch}{#1}
	\tabular
}{
	\endtabular
}
\usepackage{changepage}

% Переопределение стандартных \section, \subsection, \subsubsection по ГОСТу;
% Переопределение их отступов до и после для 1.5 интервала во всем документе
\usepackage{titlesec}

\titleformat{\section}[block]
{\bfseries\normalsize}{\thesection}{1em}{}
\titlespacing\section{\parindent}{\parskip}{\parskip}

\titleformat{\subsection}[hang]
{\bfseries\normalsize}{\thesubsection}{1em}{}
\titlespacing\subsection{\parindent}{\parskip}{\parskip}

\titleformat{\subsubsection}[hang]
{\bfseries\normalsize}{\thesubsubsection}{1em}{}
\titlespacing\subsubsection{\parindent}{\parskip}{\parskip}

\newcommand{\specsection}[1]{\section*{#1}\addcontentsline{toc}{section}{#1}}

% Работа с изображениями и таблицами; переопределение названий по ГОСТу
\usepackage{caption}
\captionsetup[figure]{name={Рисунок},labelsep=endash}
%\captionsetup[lstlisting]{name={Листинг}, labelsep=endash}
\captionsetup[table]{labelsep=endash, position=top, justification=RaggedLeft, singlelinecheck=off, }


\usepackage{graphicx}
\usepackage{adjustbox}
\usepackage{diagbox} % Диагональное разделение первой ячейки в таблицах

% Цвета для гиперссылок и листингов
\usepackage{color}

% Гиперссылки \toc с кликабельностью
\usepackage[linktoc=all]{hyperref}
\hypersetup{hidelinks}


\usepackage{color} % Цвета для гиперссылок и листингов
\hypersetup{citecolor=black}
\newcommand\nostringstyle[1]{\if@instring\else#1\fi}
\usepackage{listings}
\lstset{
	basicstyle=\footnotesize\ttfamily,
	language=prolog,
	numbers=left,
	numbersep=5pt,
	tabsize=2,
	extendedchars=true,
	breaklines=true,
	keywordstyle=\color{blue},
	frame=single,
	inputencoding=utf8x,
	showspaces=false,
	showtabs=false,
	xleftmargin=17pt,
	framexleftmargin=17pt,
	framexrightmargin=-5pt,
	framexbottommargin=4pt,
	showstringspaces=false,
	keepspaces=true,
	commentstyle={},
	escapeinside=§§
}
\newcommand{\commentfont}{\ttfamily}

\captionsetup[lstlisting]{
	singlelinecheck=false,
	margin=0pt,
	labelsep=endash
}

\usepackage{ulem} % Нормальное нижнее подчеркивание
\usepackage[figure,table]{totalcount} % Подсчет изображений, таблиц
\usepackage{rotating} % Поворот изображения вместе с названием
\usepackage{lastpage} % Для подсчета числа страниц

\makeatletter
\renewcommand\@biblabel[1]{#1.}
\makeatother

\newcommand{\anonsection}[1]{%
	\section*{#1}%
	\addcontentsline{toc}{section}{#1}%
}

\usepackage{amsmath}
\usepackage{slashbox}
\usepackage{pgfplots}
\pgfplotsset{width=0.9\linewidth,compat=1.18} 
